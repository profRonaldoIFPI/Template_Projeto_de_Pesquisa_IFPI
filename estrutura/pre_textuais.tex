% ========================================================================
% ELEMENTOS PRÉ-TEXTUAIS
% Capa, folha de rosto, listas e sumário conforme ABNT NBR 14724:2011
% ========================================================================

% ========================================================================
% CAPA
% ========================================================================
\imprimircapa

% ========================================================================
% FOLHA DE ROSTO
% ========================================================================
\imprimirfolhaderosto*

% ========================================================================
% FOLHA DE APROVAÇÃO (OPCIONAL)
% ========================================================================
% Descomente as linhas abaixo se necessário incluir folha de aprovação
% \begin{folhadeaprovacao}
%     \begin{center}
%         {\ABNTEXchapterfont\large\imprimirautor}
%         
%         \vspace*{\fill}\vspace*{\fill}
%         
%         {\ABNTEXchapterfont\bfseries\Large\imprimirtitulo}
%         
%         \vspace*{\fill}
%         
%         \hspace{.45\textwidth}
%         \begin{minipage}{.5\textwidth}
%             \imprimirpreambulo
%         \end{minipage}%
%         
%         \vspace*{\fill}
%         
%         Trabalho aprovado. \imprimirlocal, \imprimirdata:
%         
%         \vspace*{1cm}
%         
%         \assinatura{\textbf{\imprimirorientador} \\ Orientador}
%         
%         \assinatura{\textbf{Professor} \\ Convidado 1}
%         
%         \assinatura{\textbf{Professor} \\ Convidado 2}
%         
%         \vspace*{\fill}
%         
%         {\large\imprimirlocal}
%         \par
%         {\large\imprimirdata}
%         \vspace*{1cm}
%     \end{center}
% \end{folhadeaprovacao}

% ========================================================================
% DEDICATÓRIA (OPCIONAL)
% ========================================================================
% Descomente e personalize se desejar incluir dedicatória
% \begin{dedicatoria}
%     \vspace*{\fill}
%     \centering
%     \noindent
%     \textit{Dedico este trabalho aos meus pais, \\
%     pelo apoio incondicional em todos os momentos.} \vspace*{\fill}
% \end{dedicatoria}

% ========================================================================
% AGRADECIMENTOS (OPCIONAL)
% ========================================================================
% Descomente e personalize se desejar incluir agradecimentos
% \begin{agradecimentos}
%     Agradeço primeiramente a Deus, por ter me dado força e determinação para concluir este trabalho.
%     
%     Ao meu orientador, Prof. Dr. [Nome], pela paciência, dedicação e conhecimentos transmitidos durante todo o processo de orientação.
%     
%     Aos professores do Instituto Federal do Piauí -- Campus Floriano, pelos ensinamentos e contribuições para minha formação acadêmica.
%     
%     Aos meus familiares e amigos, pelo apoio e compreensão durante esta jornada.
%     
%     A todos que, direta ou indiretamente, contribuíram para a realização deste trabalho.
% \end{agradecimentos}

% ========================================================================
% EPÍGRAFE (OPCIONAL)
% ========================================================================
% Descomente e personalize se desejar incluir epígrafe
% \begin{epigrafe}
%     \vspace*{\fill}
%     \begin{flushright}
%         \textit{``A educação é a arma mais poderosa \\
%         que você pode usar para mudar o mundo.''\\
%         (Nelson Mandela)}
%     \end{flushright}
% \end{epigrafe}

% ========================================================================
% RESUMO EM PORTUGUÊS
% ========================================================================
\setlength{\absparsep}{18pt} % Ajusta espaçamento entre parágrafos do resumo

\begin{resumo}
    Este projeto de pesquisa tem como objetivo [DESCREVER O OBJETIVO PRINCIPAL DA PESQUISA]. A metodologia utilizada baseia-se em [DESCREVER A METODOLOGIA: pesquisa bibliográfica, estudo de caso, pesquisa experimental, etc.]. Como referencial teórico, serão utilizados os trabalhos de [CITAR PRINCIPAIS AUTORES DA ÁREA]. A pesquisa justifica-se pela [APRESENTAR A JUSTIFICATIVA E RELEVÂNCIA DO TEMA]. Os resultados esperados incluem [DESCREVER OS RESULTADOS ESPERADOS]. Este estudo contribuirá para [APRESENTAR A CONTRIBUIÇÃO ESPERADA PARA A ÁREA DE CONHECIMENTO]. A pesquisa será desenvolvida no período de [PERÍODO DE EXECUÇÃO] e seguirá as normas éticas estabelecidas para pesquisas acadêmicas.
    
    \textbf{Palavras-chave}: Palavra-chave 1. Palavra-chave 2. Palavra-chave 3. Palavra-chave 4. Palavra-chave 5.
\end{resumo}

% ========================================================================
% RESUMO EM INGLÊS (ABSTRACT)
% ========================================================================
\begin{resumo}[Abstract]
    \begin{otherlanguage*}{english}
        This research project aims to [DESCRIBE THE MAIN OBJECTIVE OF THE RESEARCH]. The methodology used is based on [DESCRIBE THE METHODOLOGY: bibliographic research, case study, experimental research, etc.]. As theoretical framework, the works of [CITE MAIN AUTHORS IN THE FIELD] will be used. The research is justified by [PRESENT THE JUSTIFICATION AND RELEVANCE OF THE THEME]. Expected results include [DESCRIBE EXPECTED RESULTS]. This study will contribute to [PRESENT THE EXPECTED CONTRIBUTION TO THE KNOWLEDGE AREA]. The research will be developed in the period of [EXECUTION PERIOD] and will follow the ethical standards established for academic research.
        
        \textbf{Keywords}: Keyword 1. Keyword 2. Keyword 3. Keyword 4. Keyword 5.
    \end{otherlanguage*}
\end{resumo}

% ========================================================================
% LISTA DE ILUSTRAÇÕES
% ========================================================================
% LISTA DE ILUSTRACOES
% ========================================================================
% A lista sera gerada automaticamente se houver figuras no documento
\pdfbookmark[0]{\listfigurename}{lof}
\listoffigures*
\cleardoublepage  % quebra de página

% ========================================================================
% LISTA DE TABELAS
% ========================================================================
% A lista sera gerada automaticamente se houver tabelas no documento
\pdfbookmark[0]{\listtablename}{lot}
\listoftables*
\cleardoublepage

% ========================================================================
% LISTA DE ABREVIATURAS E SIGLAS
% ========================================================================
% Descomente e adicione as abreviaturas utilizadas no trabalho
% \begin{siglas}
%     \item[ABNT] Associação Brasileira de Normas Técnicas
%     \item[IFPI] Instituto Federal de Educação, Ciência e Tecnologia do Piauí
%     \item[NBR] Norma Brasileira
%     \item[PDF] Portable Document Format
%     \item[TCC] Trabalho de Conclusão de Curso
%     % Adicione outras siglas conforme necessário
%     % \item[SIGLA] Significado da sigla
% \end{siglas}

% ========================================================================
% LISTA DE SÍMBOLOS
% ========================================================================
% Descomente e adicione os símbolos utilizados no trabalho
% \begin{simbolos}
%     \item[$ \Gamma $] Letra grega Gama
%     \item[$ \Lambda $] Lambda
%     \item[$ \zeta $] Letra grega minúscula zeta
%     \item[$ \in $] Pertence
%     % Adicione outros símbolos conforme necessário
% \end{simbolos}

% ========================================================================
% SUMÁRIO
% ========================================================================
\pdfbookmark[0]{\contentsname}{toc}
\tableofcontents*
% \cleardoublepage

% ========================================================================
% CONFIGURAÇÕES PARA INÍCIO DO TEXTO
% ========================================================================
% Configura numeração de páginas para o texto principal
\textual

% ========================================================================
% INSTRUÇÕES PARA PERSONALIZAÇÃO
% ========================================================================
%
% COMO PERSONALIZAR ESTE ARQUIVO:
%
% 1. RESUMO: Substitua o texto de exemplo pelo resumo real do seu projeto
%    - Máximo de 500 palavras
%    - Deve conter: objetivo, metodologia, resultados esperados
%    - Inclua 3-5 palavras-chave relevantes
%
% 2. ABSTRACT: Traduza o resumo para o inglês
%    - Mantenha a mesma estrutura do resumo em português
%    - Use keywords correspondentes às palavras-chave
%
% 3. LISTAS AUTOMÁTICAS:
%    - Lista de figuras: gerada automaticamente quando há figuras
%    - Lista de tabelas: gerada automaticamente quando há tabelas
%    - Para não aparecer listas vazias, comente as linhas correspondentes
%
% 4. LISTA DE SIGLAS:
%    - Adicione todas as siglas utilizadas no trabalho
%    - Mantenha ordem alfabética
%    - Use o formato: \item[SIGLA] Significado completo
%
% 5. ELEMENTOS OPCIONAIS:
%    - Dedicatória: descomente e personalize se desejar
%    - Agradecimentos: descomente e personalize se desejar
%    - Epígrafe: descomente e personalize se desejar
%    - Folha de aprovação: descomente se necessário
%
% 6. LISTA DE SÍMBOLOS:
%    - Descomente se o trabalho utilizar símbolos matemáticos
%    - Adicione os símbolos na ordem de aparição no texto
%
% ========================================================================
