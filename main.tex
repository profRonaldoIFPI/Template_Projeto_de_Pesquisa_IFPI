% ========================================================================
% TEMPLATE LATEX PARA PROJETOS DE PESQUISA - IFPI
% Baseado na classe abntex2 com personalizações para o IFPI
% Conforme e Manual de Normalização de Trabalhos Acadêmicos do IFPI - 2024 
% e as normas ABNT NBR:
%   14724:2011
%   6024:2012
%   6027:2012
%   10520:2023
%   6023:2018 
%   15287:2011 
% ========================================================================

% ========================================================================
% PREÂMBULO E CONFIGURAÇÕES
% ========================================================================

% ========================================================================
% CLASSE DO DOCUMENTO
% ========================================================================
\documentclass[
    12pt,        % Tamanho da fonte principal
    openany,     % Capitulos podem iniciar em qualquer pagina
    oneside,     % Impressao apenas frente
    a4paper,     % Formato A4
    english,     % Idioma adicional
    french,      % Idioma adicional
    spanish,     % Idioma adicional
    brazil       % Idioma principal
]{abntex2}

% ========================================================================
% PACOTES FUNDAMENTAIS
% ========================================================================

% Codificacao e fontes
\usepackage[utf8]{inputenc}
\usepackage[T1]{fontenc}
\usepackage{lmodern}
\usepackage{times}

% Idioma
\usepackage[brazil]{babel}

% Regras ABNT / referencias
\usepackage[alf]{abntex2cite}
\usepackage{config/abntex-ifpi}

% ========================================================================
% PACOTES PARA FORMATACAO E LAYOUT
% ========================================================================
\usepackage{microtype}
\usepackage{indentfirst}
\usepackage{color}
\usepackage{graphicx}
\usepackage{float}
\usepackage{amsmath,amsfonts,amssymb}
\usepackage{booktabs}
\usepackage{array}
\usepackage{longtable}
\usepackage{multirow}
\usepackage{rotating}

% Listas, siglas e simbolos
\usepackage{enumitem}
\usepackage{acronym}
\usepackage{nomencl}

% URLs
\usepackage{url}

% ========================================================================
% MARGENS, ESPACAMENTO E NUMERACAO
% ========================================================================
\setlrmarginsandblock{3cm}{2cm}{*}
\setulmarginsandblock{3cm}{2cm}{*}
\checkandfixthelayout

\OnehalfSpacing
\setlength{\parindent}{1.25cm}

\setcounter{secnumdepth}{3}
\setcounter{tocdepth}{3}

% ========================================================================
% LISTAS E NOMES PADRAO
% ========================================================================
\renewcommand{\listfigurename}{LISTA DE ILUSTRACOES}
\renewcommand{\listtablename}{LISTA DE TABELAS}
\renewcommand{\nomname}{LISTA DE ABREVIATURAS E SIGLAS}

% ========================================================================
% LEGENDAS E AUTOREF
% ========================================================================
\renewcommand{\figureautorefname}{Figura}
\renewcommand{\tableautorefname}{Tabela}
\renewcommand{\equationautorefname}{Equacao}
\renewcommand{\quadroautorefname}{Quadro}
% ========================================================================
% COMANDOS UTILITARIOS
% ========================================================================
\newcommand{\citep}[2]{\cite[p.~#2]{#1}}
\newcommand{\citeonlinep}[2]{\citeonline[p.~#2]{#1}}
\newcommand{\citemulti}[1]{\cite{#1}}

\newcommand{\inserirfigura}[4]{%
    \begin{figure}[htbp]
        \centering
        \caption{#3}
        \includegraphics[width=#2]{#1}
        \label{fig:#4}
        \fontetabela{Elaborado pelo autor.}
    \end{figure}
}

\newcommand{\fontetabela}[1]{%
    % \vspace{0.2cm}
    \begin{center}
        {\footnotesize Fonte: #1}
    \end{center}
}

% ========================================================================
% HYPERREF (adiado para depois dos dados do trabalho)
% ========================================================================
\AtBeginDocument{%
    \hypersetup{%
        pdftitle={\imprimirtitulo},
        pdfauthor={\imprimirautor},
        pdfsubject={\imprimirpreambulo},
        pdfcreator={LaTeX with abnTeX2},
        pdfkeywords={abnt}{latex}{abntex}{abntex2}{projeto de pesquisa}{IFPI},
        colorlinks=false,
        hidelinks=true,
        linkcolor=black,
        citecolor=black,
        filecolor=black,
        urlcolor=black,
    }%
}

% ========================================================================
% AJUSTES FINAIS
% ========================================================================
\sloppy
\hyphenpenalty=5000
\tolerance=1000

\setlist{nosep}
\renewcommand{\thepage}{\arabic{page}}
   %configurações de formatação

% ========================================================================
% DADOS DO PROJETO DE PESQUISA
% Edite os valores abaixo para personalizar o template.
% ========================================================================

% Instituicao e identificacao do curso
\instituicao{Instituto Federal de Educacao, Ciencia e Tecnologia do Piaui}
\campus{Campus Floriano}
\curso{Tecnologia em Analise e Desenvolvimento de Sistemas}
\areaconcentracao{} % Opcional: defina a area de concentracao ou deixe vazio

% Titulo e autoria
\titulo{TITULO DO PROJETO DE PESQUISA: SUBTITULO SE HOUVER}
\autor{NOME COMPLETO DO AUTOR}

% Orientacao (comente ou esvazie o comando se nao houver coorientador)
\orientador{Prof. Me. Ronaldo Pires Borges}
\coorientador{} % Exemplo: Prof. Dr. Nome do Coorientador

% Informacoes gerais
\local{Floriano -- PI}
\data{2025}

% Descricao apresentada na folha de rosto
\preambulo{Projeto de pesquisa apresentado ao Instituto Federal de Educacao, Ciencia e Tecnologia do Piaui -- Campus Floriano, como requisito para aprovação na disciplina de Projeto de Pesquisa do curso de Tecnologia em Analise e Desenvolvimento de Sistemas.}

% Natureza do trabalho 
% \naturezatrabalho{Projeto de pesquisa apresentado ao Instituto Federal de Educacao, Ciencia e Tecnologia do Piaui como requisito para aprovação na disciplina...}

% Grau pretendido (opcional)
% \grau{} 
% \grau{Tecnólogo em Analise e Desenvolvimento de Sistemas} 

% ========================================================================
% INSTRUCOES RAPIDAS
% - Substitua os textos em caixa alta pelos dados reais.
% - Campos vazios podem ser removidos ou comentados.
% - Para atualizar o resumo e demais secoes, edite os arquivos em principal.tex.
% ========================================================================
           %dados institucionais e do projeto

% ========================================================================
% INÍCIO DO DOCUMENTO
% ========================================================================

\begin{document}
% Seleciona o idioma do documento (conforme pacotes do babel)
\selectlanguage{brazil}
% Retira espaço extra obsoleto entre as frases
\frenchspacing

% ========================================================================
% ELEMENTOS PRÉ-TEXTUAIS
% ========================================================================
% ========================================================================
% ELEMENTOS PRÉ-TEXTUAIS
% Capa, folha de rosto, listas e sumário conforme ABNT NBR 14724:2011
% ========================================================================

% ========================================================================
% CAPA
% ========================================================================
\imprimircapa

% ========================================================================
% FOLHA DE ROSTO
% ========================================================================
\imprimirfolhaderosto*

% ========================================================================
% FOLHA DE APROVAÇÃO (OPCIONAL)
% ========================================================================
% Descomente as linhas abaixo se necessário incluir folha de aprovação
% \begin{folhadeaprovacao}
%     \begin{center}
%         {\ABNTEXchapterfont\large\imprimirautor}
%         
%         \vspace*{\fill}\vspace*{\fill}
%         
%         {\ABNTEXchapterfont\bfseries\Large\imprimirtitulo}
%         
%         \vspace*{\fill}
%         
%         \hspace{.45\textwidth}
%         \begin{minipage}{.5\textwidth}
%             \imprimirpreambulo
%         \end{minipage}%
%         
%         \vspace*{\fill}
%         
%         Trabalho aprovado. \imprimirlocal, \imprimirdata:
%         
%         \vspace*{1cm}
%         
%         \assinatura{\textbf{\imprimirorientador} \\ Orientador}
%         
%         \assinatura{\textbf{Professor} \\ Convidado 1}
%         
%         \assinatura{\textbf{Professor} \\ Convidado 2}
%         
%         \vspace*{\fill}
%         
%         {\large\imprimirlocal}
%         \par
%         {\large\imprimirdata}
%         \vspace*{1cm}
%     \end{center}
% \end{folhadeaprovacao}

% ========================================================================
% DEDICATÓRIA (OPCIONAL)
% ========================================================================
% Descomente e personalize se desejar incluir dedicatória
% \begin{dedicatoria}
%     \vspace*{\fill}
%     \centering
%     \noindent
%     \textit{Dedico este trabalho aos meus pais, \\
%     pelo apoio incondicional em todos os momentos.} \vspace*{\fill}
% \end{dedicatoria}

% ========================================================================
% AGRADECIMENTOS (OPCIONAL)
% ========================================================================
% Descomente e personalize se desejar incluir agradecimentos
% \begin{agradecimentos}
%     Agradeço primeiramente a Deus, por ter me dado força e determinação para concluir este trabalho.
%     
%     Ao meu orientador, Prof. Dr. [Nome], pela paciência, dedicação e conhecimentos transmitidos durante todo o processo de orientação.
%     
%     Aos professores do Instituto Federal do Piauí -- Campus Floriano, pelos ensinamentos e contribuições para minha formação acadêmica.
%     
%     Aos meus familiares e amigos, pelo apoio e compreensão durante esta jornada.
%     
%     A todos que, direta ou indiretamente, contribuíram para a realização deste trabalho.
% \end{agradecimentos}

% ========================================================================
% EPÍGRAFE (OPCIONAL)
% ========================================================================
% Descomente e personalize se desejar incluir epígrafe
% \begin{epigrafe}
%     \vspace*{\fill}
%     \begin{flushright}
%         \textit{``A educação é a arma mais poderosa \\
%         que você pode usar para mudar o mundo.''\\
%         (Nelson Mandela)}
%     \end{flushright}
% \end{epigrafe}

% ========================================================================
% RESUMO EM PORTUGUÊS
% ========================================================================
\setlength{\absparsep}{18pt} % Ajusta espaçamento entre parágrafos do resumo

\begin{resumo}
    Este projeto de pesquisa tem como objetivo [DESCREVER O OBJETIVO PRINCIPAL DA PESQUISA]. A metodologia utilizada baseia-se em [DESCREVER A METODOLOGIA: pesquisa bibliográfica, estudo de caso, pesquisa experimental, etc.]. Como referencial teórico, serão utilizados os trabalhos de [CITAR PRINCIPAIS AUTORES DA ÁREA]. A pesquisa justifica-se pela [APRESENTAR A JUSTIFICATIVA E RELEVÂNCIA DO TEMA]. Os resultados esperados incluem [DESCREVER OS RESULTADOS ESPERADOS]. Este estudo contribuirá para [APRESENTAR A CONTRIBUIÇÃO ESPERADA PARA A ÁREA DE CONHECIMENTO]. A pesquisa será desenvolvida no período de [PERÍODO DE EXECUÇÃO] e seguirá as normas éticas estabelecidas para pesquisas acadêmicas.
    
    \textbf{Palavras-chave}: Palavra-chave 1. Palavra-chave 2. Palavra-chave 3. Palavra-chave 4. Palavra-chave 5.
\end{resumo}

% ========================================================================
% RESUMO EM INGLÊS (ABSTRACT)
% ========================================================================
\begin{resumo}[Abstract]
    \begin{otherlanguage*}{english}
        This research project aims to [DESCRIBE THE MAIN OBJECTIVE OF THE RESEARCH]. The methodology used is based on [DESCRIBE THE METHODOLOGY: bibliographic research, case study, experimental research, etc.]. As theoretical framework, the works of [CITE MAIN AUTHORS IN THE FIELD] will be used. The research is justified by [PRESENT THE JUSTIFICATION AND RELEVANCE OF THE THEME]. Expected results include [DESCRIBE EXPECTED RESULTS]. This study will contribute to [PRESENT THE EXPECTED CONTRIBUTION TO THE KNOWLEDGE AREA]. The research will be developed in the period of [EXECUTION PERIOD] and will follow the ethical standards established for academic research.
        
        \textbf{Keywords}: Keyword 1. Keyword 2. Keyword 3. Keyword 4. Keyword 5.
    \end{otherlanguage*}
\end{resumo}

% ========================================================================
% LISTA DE ILUSTRAÇÕES
% ========================================================================
% LISTA DE ILUSTRACOES
% ========================================================================
% A lista sera gerada automaticamente se houver figuras no documento
\pdfbookmark[0]{\listfigurename}{lof}
\listoffigures*
\cleardoublepage  % quebra de página

% ========================================================================
% LISTA DE TABELAS
% ========================================================================
% A lista sera gerada automaticamente se houver tabelas no documento
\pdfbookmark[0]{\listtablename}{lot}
\listoftables*
\cleardoublepage

% ========================================================================
% LISTA DE ABREVIATURAS E SIGLAS
% ========================================================================
% Descomente e adicione as abreviaturas utilizadas no trabalho
% \begin{siglas}
%     \item[ABNT] Associação Brasileira de Normas Técnicas
%     \item[IFPI] Instituto Federal de Educação, Ciência e Tecnologia do Piauí
%     \item[NBR] Norma Brasileira
%     \item[PDF] Portable Document Format
%     \item[TCC] Trabalho de Conclusão de Curso
%     % Adicione outras siglas conforme necessário
%     % \item[SIGLA] Significado da sigla
% \end{siglas}

% ========================================================================
% LISTA DE SÍMBOLOS
% ========================================================================
% Descomente e adicione os símbolos utilizados no trabalho
% \begin{simbolos}
%     \item[$ \Gamma $] Letra grega Gama
%     \item[$ \Lambda $] Lambda
%     \item[$ \zeta $] Letra grega minúscula zeta
%     \item[$ \in $] Pertence
%     % Adicione outros símbolos conforme necessário
% \end{simbolos}

% ========================================================================
% SUMÁRIO
% ========================================================================
\pdfbookmark[0]{\contentsname}{toc}
\tableofcontents*
% \cleardoublepage

% ========================================================================
% CONFIGURAÇÕES PARA INÍCIO DO TEXTO
% ========================================================================
% Configura numeração de páginas para o texto principal
\textual

% ========================================================================
% INSTRUÇÕES PARA PERSONALIZAÇÃO
% ========================================================================
%
% COMO PERSONALIZAR ESTE ARQUIVO:
%
% 1. RESUMO: Substitua o texto de exemplo pelo resumo real do seu projeto
%    - Máximo de 500 palavras
%    - Deve conter: objetivo, metodologia, resultados esperados
%    - Inclua 3-5 palavras-chave relevantes
%
% 2. ABSTRACT: Traduza o resumo para o inglês
%    - Mantenha a mesma estrutura do resumo em português
%    - Use keywords correspondentes às palavras-chave
%
% 3. LISTAS AUTOMÁTICAS:
%    - Lista de figuras: gerada automaticamente quando há figuras
%    - Lista de tabelas: gerada automaticamente quando há tabelas
%    - Para não aparecer listas vazias, comente as linhas correspondentes
%
% 4. LISTA DE SIGLAS:
%    - Adicione todas as siglas utilizadas no trabalho
%    - Mantenha ordem alfabética
%    - Use o formato: \item[SIGLA] Significado completo
%
% 5. ELEMENTOS OPCIONAIS:
%    - Dedicatória: descomente e personalize se desejar
%    - Agradecimentos: descomente e personalize se desejar
%    - Epígrafe: descomente e personalize se desejar
%    - Folha de aprovação: descomente se necessário
%
% 6. LISTA DE SÍMBOLOS:
%    - Descomente se o trabalho utilizar símbolos matemáticos
%    - Adicione os símbolos na ordem de aparição no texto
%
% ========================================================================


% ========================================================================
% ELEMENTOS TEXTUAIS
% ========================================================================

\chapter{INTRODUÇÃO}
\label{chap:introducao}

A introdução de um projeto de pesquisa deve apresentar o tema de forma clara e contextualizada, demonstrando sua relevância no cenário atual. Este capítulo tem como objetivo situar o leitor no contexto da pesquisa, apresentando o problema a ser investigado e justificando a importância do estudo.

O tema desta pesquisa insere-se no campo de [ÁREA DE CONHECIMENTO], especificamente abordando [TEMA ESPECÍFICO]. A escolha deste tema justifica-se pela [JUSTIFICATIVA INICIAL], considerando sua relevância tanto no âmbito acadêmico quanto na aplicação prática.

Segundo \citeonline{marconi2017}, a área de [ÁREA DE CONHECIMENTO] tem apresentado crescente interesse por estudos que abordem [ASPECTO RELEVANTE DO TEMA]. Esta tendência evidencia a necessidade de pesquisas que contribuam para o avanço do conhecimento nesta área.

O presente projeto de pesquisa propõe investigar [OBJETO DE ESTUDO], com foco em [ASPECTO ESPECÍFICO A SER INVESTIGADO]. A pesquisa será desenvolvida considerando [CONTEXTO OU DELIMITAÇÃO DO ESTUDO], o que permitirá uma análise aprofundada do fenômeno em questão.


\section{PROBLEMA DE PESQUISA}
\label{sec:problema}

O problema de pesquisa constitui o elemento central que orienta todo o desenvolvimento do estudo. Conforme \citeonline{gil2019}, um problema bem formulado deve ser claro, preciso e operacional, permitindo a definição de objetivos específicos e a escolha de métodos adequados para sua investigação.

No contexto desta pesquisa, observa-se que [DESCRIÇÃO DO PROBLEMA IDENTIFICADO]. Esta situação tem gerado [CONSEQUÊNCIAS OU IMPACTOS DO PROBLEMA], evidenciando a necessidade de estudos que possam contribuir para sua compreensão e possível solução.

Diante do exposto, formula-se o seguinte problema de pesquisa:

\textbf{[PERGUNTA DE PESQUISA PRINCIPAL]?}

Este questionamento surge da observação de que [CONTEXTUALIZAÇÃO DO PROBLEMA], conforme evidenciado por \citeonline{silva2023} em seus estudos sobre [ÁREA RELACIONADA].



\section{HIPÓTESES}
\label{sec:hipoteses}   

As hipóteses representam respostas provisórias ao problema de pesquisa, baseadas no conhecimento teórico disponível e na experiência do pesquisador. Segundo \citeonline{marconi2017}, as hipóteses devem ser testáveis e formuladas de maneira clara e precisa.

Para o problema apresentado, formulam-se as seguintes hipóteses:


\textbf{Hipótese Principal (H1):} [HIPÓTESE PRINCIPAL DA PESQUISA]

\textbf{Hipóteses Secundárias:}
\begin{itemize}
    \item \textbf{H2:} [SEGUNDA HIPÓTESE]
    \item \textbf{H3:} [TERCEIRA HIPÓTESE, SE APLICÁVEL]
\end{itemize}

Essas hipóteses serão testadas através da metodologia proposta, permitindo confirmar ou refutar as suposições iniciais da pesquisa.



\section{OBJETIVOS}
\label{sec:objetivos}

Os objetivos definem o que se pretende alcançar com a realização da pesquisa. Devem ser formulados de forma clara e precisa, utilizando verbos no infinitivo que expressem ações concretas e mensuráveis.

\subsection{Objetivo Geral}
\label{subsec:objetivo-geral}

[DESCREVER O OBJETIVO GERAL DA PESQUISA, UTILIZANDO VERBO NO INFINITIVO E EXPRESSANDO DE FORMA CLARA O QUE SE PRETENDE ALCANÇAR COM O ESTUDO]

\subsection{Objetivos Específicos}
\label{subsec:objetivos-especificos}

\begin{enumerate}
    \item Objetivo específico 1;
    \item Objetivo específico 2;
    \item Objetivo específico 3;
    \item Objetivo específico 4 (se aplicável).
\end{enumerate}

\section{JUSTIFICATIVA}
\label{sec:justificativa}

A justificativa deve demonstrar a relevância e a importância da pesquisa, apresentando argumentos que evidenciem sua contribuição para o avanço do conhecimento científico e/ou para a solução de problemas práticos.

Esta pesquisa justifica-se por diversos aspectos:

\textbf{Relevância Científica:} O tema abordado contribui para o avanço do conhecimento na área de [ÁREA DE CONHECIMENTO], especialmente no que se refere a [ASPECTO ESPECÍFICO]. Conforme destacado por \citeonline{inep2022}, existe uma lacuna na literatura sobre [LACUNA IDENTIFICADA], que esta pesquisa pretende contribuir para preencher.

\textbf{Relevância Social:} Os resultados desta pesquisa poderão beneficiar [PÚBLICO-ALVO OU SOCIEDADE EM GERAL], uma vez que [EXPLICAR COMO OS RESULTADOS PODEM SER APLICADOS]. Segundo dados de \cite{instituicao2023}, [APRESENTAR DADOS QUE EVIDENCIEM A RELEVÂNCIA SOCIAL].

\textbf{Relevância Acadêmica:} Este estudo contribuirá para a formação acadêmica do pesquisador e para o desenvolvimento de pesquisas futuras na área, fornecendo subsídios teóricos e metodológicos para estudos similares.

\textbf{Viabilidade:} A pesquisa é viável considerando os recursos disponíveis, o tempo previsto para sua execução e o acesso às fontes de informação necessárias.

\chapter{REFERENCIAL TEÓRICO}
\label{chap:referencial}

O referencial teórico constitui a base conceitual da pesquisa, apresentando as teorias, conceitos e estudos que fundamentam a investigação. Este capítulo deve demonstrar o conhecimento do pesquisador sobre o tema e situar a pesquisa no contexto do conhecimento científico existente.

\section{[TÍTULO DA PRIMEIRA SEÇÃO TEÓRICA]}
\label{sec:fundamentacao1}

[DESENVOLVER A FUNDAMENTAÇÃO TEÓRICA PRINCIPAL DO TEMA, APRESENTANDO OS CONCEITOS CENTRAIS E AS TEORIAS QUE EMBASAM A PESQUISA]

Segundo \citeonline{creswell2018}, [APRESENTAR CONCEITO OU TEORIA FUNDAMENTAL]. Esta definição é complementada por \citeonline{gil2019}, que acrescenta [INFORMAÇÃO COMPLEMENTAR].

\begin{figure}[htbp]
    \centering
    \caption{Exemplo de figura no documento}
    \includegraphics[width=0.3\textwidth]{img/Logo-IFPI-Floriano-Horizontal.png}
    \fonte{IFPI - Instituto Federal do Piauí.}
    \label{fig:exemplo}
\end{figure}

Quando elementos visuais, como figuras, quadros ou tabelas, forem inseridos no texto, devem ser obrigatoriamente identificados e mencionados no corpo do trabalho, de modo a estabelecer uma relação clara entre a informação apresentada e sua discussão. Conforme orienta a \citeonline{ABNT_NBR14724_2011},  esses elementos devem ser citados no texto por seu número e título, garantindo uniformidade e rastreabilidade. Assim, pode-se fazer referência, por exemplo, à \autoref{fig:exemplo}, na qual se apresenta a logomarca horizontal do IFPI Campus Floriano.

% ou
% \inserirfigura{img/Logo-IFPI-Floriano-Horizontal.png}{0.3\textwidth}{Exemplo de figura no documento}{exemplo}

% ========================================================================
% CITAÇÃO DIRETA LONGA - EXEMPLO
% ========================================================================
% Citação direta com mais de 3 linhas deve ser destacada com recuo de 4 cm da margem esquerda,
% sem aspas, com fonte menor (tamanho 10) e espaçamento simples.
% A formatação automática do abntex2 já aplica estas regras quando se usa o ambiente citacao.

\begin{citacao}
Quando uma citação direta ultrapassa três linhas do texto original, ela deve ser apresentada em destaque, com recuo especial da margem esquerda, sem o uso de aspas, utilizando fonte menor que a do corpo do texto e espaçamento simples entre as linhas. Esta formatação diferenciada permite ao leitor identificar claramente que se trata de um trecho literalmente reproduzido de outra fonte, mantendo a integridade do pensimento do autor ciado \cite{ABNT_NBR10520_2023}.
\end{citacao}

\section{ESTUDOS RELACIONADOS}
\label{sec:estudos-relacionados} 

Esta seção apresenta uma revisão dos principais estudos relacionados ao tema da pesquisa, evidenciando o estado da arte e identificando lacunas que justifiquem a realização do presente estudo.

A \textbf{tabela} é uma forma não discursiva de apresentação de informações em que o dado numérico é o elemento central, organizada de modo a facilitar comparações, análises e leitura rápida dos resultados. Segundo as normas da ABNT, as tabelas devem apresentar título centralizado, estrutura aberta (sem bordas laterais ou inferiores fechadas), fonte abaixo e devem ser citadas no texto conforme sua numeração \cite{ABNT_NBR14724_2011,ABNT_NBR15287_2011,ABNT_NBR6024_2012}. Lembre-se que toda tabela, figura ou quadro inseridos no seu trabalhos dever ser citatdo no texto, como por exemplo, a \autoref{tab:estudos}, que exemplifica a formatação adequada de uma tabela.

\begin{table}[htbp] %"table" é o ambiente padrão do LaTeX para tabelas.
    \centering
    \caption{Exemplo de tabela }
    \label{tab:estudos}
    \begin{tabular}{p{4cm}cc}
        \hline
        \textbf{Estudo} & \textbf{Amostra} & \textbf{Métrica principal} \\ %\\ indica quebra de linha
        \hline\hline %Linha dupla para separar o cabeçalho do conteúdo
        Costa e Rocha (2023) & 120 respondentes & Média de adoção = 4,1 \\
        Silva e Santos (2023) & 18 escolas analisadas & Taxa de conclusão = 87\% \\
        Souza (2021) & 32 entrevistas & Índice de concordância = 0,74 \\
        \hline
    \end{tabular}
    \fonte{Elaborado pelo autor a partir dos estudos revisados.}
\end{table}

% =============== OBSERVAÇÃO SOBRE POSICIONAMENTO DE TABELAS E QUADROS ==============
% No ambiente table (tabela) e quadro (quadro), o LaTeX permite especificar opções de posicionamento.

% \begin{table}[htbp] <- Essas letras são indicadores de posicionamento:

% h	here = aqui	“coloque exatamente aqui se possível”
% t	top = topo	“se não couber aqui, tente no topo da página”
% b	bottom = rodapé	“ou no fim da página”
% p	page of floats	“ou faça uma página só de tabelas/figuras”

% Assim, [htbp] significa:

% “Tente colocar aqui; se não der, tenta no topo; depois no rodapé; depois numa página separada.”
% É o mais usado porque dá flexibilidade, o que evita que o LaTeX “quebre” seu layout.
% ====================================================================================

O \textbf{quadro}, diferentemente da tabela, é utilizado para apresentar dados qualitativos, como definições, características, categorias ou descrições. Ele possui estrutura fechada em todos os lados e organiza informações textuais de maneira comparativa e clara, sendo recomendado para sínteses conceituais ou classificações dentro do projeto de pesquisa \cite{ABNT_NBR14724_2011,ABNT_NBR15287_2011,ABNT_NBR6024_2012}. O \autoref{qua:estudos} exemplifica a formatação adequada de um quadro.

\begin{quadro}[htbp]  %"quadro" é o ambiente personalizado do abntex2 para quadros
    \centering
    \caption{Exemplo de quadro}
    \label{qua:estudos}
    \begin{tabular}{|p{3cm}|p{4cm}|p{6cm}|} % os caracteres | indicam as bordas verticais
        \hline
        \textbf{Autor/Ano} & \textbf{Metodologia} & \textbf{Principais Resultados} \\
        \hline
        Autor A (2023) & Pesquisa quantitativa & Ênfase na mensuração de impacto tecnológico \\ \hline
        Autor B (2022) & Estudo de caso & Identifica barreiras institucionais para inovação \\ \hline
        Autor C (2021) & Pesquisa qualitativa & Discute competências desenvolvidas pelos participantes \\ \hline
    \end{tabular}
    \fonte{Elaborado pelo autor.}
\end{quadro}

Você pode usar a feramenta \textit{Tables Genetator}\footnote{https://www.tablesgenerator.com/} para criar tabelas e quadros.

\chapter{METODOLOGIA}
\label{chap:metodologia}

A metodologia descreve os procedimentos que serão adotados para a realização da pesquisa, incluindo a caracterização do estudo, os métodos de coleta e análise de dados, e os aspectos éticos envolvidos.

\section{CARACTERIZAÇÃO DA PESQUISA}
\label{sec:caracterizacao}

Esta pesquisa pode ser caracterizada, quanto aos seus objetivos, como [EXPLORATÓRIA/DESCRITIVA/EXPLICATIVA], uma vez que [JUSTIFICAR A CLASSIFICAÇÃO]. Segundo \citeonline{gil2019}, pesquisas [TIPO DA PESQUISA] têm como objetivo [OBJETIVO DO TIPO DE PESQUISA].

Quanto à abordagem, trata-se de uma pesquisa [QUALITATIVA/QUANTITATIVA/MISTA], pois [JUSTIFICAR A ABORDAGEM]. \citeonline{creswell2018} destaca que a abordagem [TIPO DE ABORDAGEM] é adequada quando [SITUAÇÕES EM QUE A ABORDAGEM É ADEQUADA].

Em relação aos procedimentos técnicos, será realizada uma [PESQUISA BIBLIOGRÁFICA/ESTUDO DE CASO/PESQUISA DE CAMPO/ETC.], que, conforme \citeonline{marconi2017}, caracteriza-se por [CARACTERÍSTICAS DO PROCEDIMENTO ESCOLHIDO].

\begin{figure}[htbp]
    \centering
    \caption{Figura que ilustre a sua metodologia}
    \includegraphics[width=0.9\textwidth]{img/tema do tcc.png}
    \fonte{Adaptado de \citeonline{Moretti_ComoEscolherTemaTCC_2024}}
    \label{fig:exemplo2}
\end{figure}

\section{POPULAÇÃO E AMOSTRA}
\label{sec:populacao-amostra}

A população desta pesquisa é constituída por [DEFINIR A POPULAÇÃO]. Para a definição da amostra, será utilizada a técnica de [TIPO DE AMOSTRAGEM], que, segundo \citeonline{hair2019}, é adequada quando [JUSTIFICAR A ESCOLHA DA TÉCNICA DE AMOSTRAGEM].

O tamanho da amostra foi calculado considerando [CRITÉRIOS PARA CÁLCULO DA AMOSTRA], resultando em [NÚMERO] participantes/elementos.

\section{COLETA DE DADOS}
\label{sec:coleta-dados}

A coleta de dados será realizada através de [INSTRUMENTOS DE COLETA], que permitirão obter informações sobre [TIPO DE INFORMAÇÕES A SEREM COLETADAS].

Os instrumentos escolhidos foram:
\begin{itemize}
    \item \textbf{[INSTRUMENTO 1]:} [DESCRIÇÃO E JUSTIFICATIVA]
    \item \textbf{[INSTRUMENTO 2]:} [DESCRIÇÃO E JUSTIFICATIVA]
\end{itemize}

\section{ANÁLISE DOS DADOS}
\label{sec:analise-dados}

Os dados coletados serão analisados utilizando [TÉCNICAS DE ANÁLISE]. Para os dados quantitativos, será utilizada [ESTATÍSTICA DESCRITIVA/INFERENCIAL], com auxílio do software [NOME DO SOFTWARE]. Os dados qualitativos serão analisados através de [TÉCNICA DE ANÁLISE QUALITATIVA].

\section{ASPECTOS ÉTICOS}
\label{sec:aspectos-eticos}

Esta pesquisa será desenvolvida respeitando os princípios éticos estabelecidos pela Resolução nº 466/2012 do Conselho Nacional de Saúde. [SE APLICÁVEL: O projeto será submetido ao Comitê de Ética em Pesquisa antes do início da coleta de dados.]

Todos os participantes serão informados sobre os objetivos da pesquisa e assinarão o Termo de Consentimento Livre e Esclarecido (TCLE), garantindo sua participação voluntária e o direito de desistir a qualquer momento.

\chapter{RESULTADOS ESPERADOS}
\label{chap:resultados-esperados}

Os resultados esperados descrevem de forma objetiva os produtos, impactos e transformações que a pesquisa pretende alcançar. Eles devem estar alinhados aos objetivos geral e específicos e demonstrar como o estudo contribuirá para a área de conhecimento e para a sociedade.

\begin{itemize}
    \item \textbf{Impactos científicos}: [APRESENTAR AS PRINCIPAIS CONTRIBUIÇÕES PARA A LITERATURA E PARA O AVANÇO TEÓRICO DA ÁREA].
    \item \textbf{Impactos tecnológicos e/ou sociais}: [DESCREVER MUDANÇAS ESPERADAS NO CONTEXTO INVESTIGADO, EM POLÍTICAS OU EM PRÁTICAS PROFISSIONAIS].
    \item \textbf{Produtos e entregáveis}: [LISTAR RELATÓRIOS, PROTÓTIPOS, SOFTWARES, ARTIGOS OU OUTROS PRODUTOS QUE SERÃO GERADOS].
    \item \textbf{Indicadores de acompanhamento}: [DEFINIR COMO OS RESULTADOS SERÃO MEDIDOS OU VERIFICADOS, MENCIONANDO MÉTRICAS OU METAS].
\end{itemize}

Para cada resultado, descreva o horizonte temporal de obtenção (curto, médio ou longo prazo) e as evidências que comprovarão seu alcance.

\chapter{RECURSOS}
\label{chap:recursos}

Esta seção identifica os recursos necessários para viabilizar a execução do projeto, conforme orientação da NBR 15287. Devem ser discriminados os recursos humanos, materiais/infraestruturais e financeiros.

\section{Recursos Humanos}
\label{sec:recursos-humanos}

[DESCREVER A EQUIPE ENVOLVIDA: PESQUISADOR PRINCIPAL, ORIENTADOR(ES), COLABORADORES, BOLSISTAS E PARCEIROS. INFORMAR AS PRINCIPAIS ATIVIDADES DE CADA PARTICIPANTE E A CARGA HORÁRIA PREVISTA.]

\section{Recursos Materiais e Infraestruturais}
\label{sec:recursos-materiais}

[DETALHAR LABORATÓRIOS, SOFTWARES, HARDWARES, INSUMOS, COLEÇÕES DE DADOS OU OUTRAS ESTRUTURAS QUE SERÃO UTILIZADAS. INFORMAR SE OS ITENS JÁ ESTÃO DISPONÍVEIS NA INSTITUIÇÃO OU SE PRECISAM SER ADQUIRIDOS/ALUGADOS.]

\subsection{Orçamento}
\label{subsec:orcamento}

O orçamento sintetiza os custos estimados e faz parte dos recursos materiais, pois evidencia os insumos financeiros necessários para assegurar os itens listados anteriormente. A \autoref{tab:orcamento} detalha os valores previstos por categoria de despesa.

\section{Recursos Financeiros}
\label{sec:recursos-financeiros}

[SE HOUVER, APRESENTAR UMA SÍNTESE DAS FONTES DE FINANCIAMENTO PREVISTAS (BOLSAS, AUXÍLIOS, CONTRAPARTIDAS INSTITUCIONAIS ETC.) E COMO OS VALORES SERÃO APLICADOS NAS DESPESAS DO PROJETO.]

Na \autoref{tab:orcamento} estão detalhados os custos financeiros de cada item necessário para a execução da pesquisa.

\begin{table}[htbp]
    \centering
    \caption{Orçamento da pesquisa}
    \label{tab:orcamento}
    \begin{tabular}{p{6cm}ccc}
        \hline
        \textbf{Item} & \textbf{Quantidade} & \textbf{Valor Unitário (R\$)} & \textbf{Valor Total (R\$)} \\
        \hline\hline
        Material de escritório & - & - & 150,00 \\
        % \hline
        Impressões e cópias & - & - & 100,00 \\
        % \hline
        Transporte & - & - & 200,00 \\
        % \hline
        Software (licenças) & - & - & 300,00 \\
        \hline
        \textbf{TOTAL} & & & \textbf{750,00} \\
        \hline
    \end{tabular}
    \fonte{Elaborado pelo autor.}
\end{table}

\chapter{CRONOGRAMA}
\label{chap:cronograma}

O cronograma apresenta a distribuição temporal das atividades previstas para a realização da pesquisa. A \autoref{qua:cronograma} detalha as etapas e seus respectivos períodos de execução.

\begin{quadro}[htbp]
    \centering
    \caption{Cronograma de execução da pesquisa}
    \label{qua:cronograma}
    \begin{tabular}{|p{6cm}|c|c|c|c|c|c|}
        \hline
        \textbf{Atividades} & \textbf{Jan} & \textbf{Fev} & \textbf{Mar} & \textbf{Abr} & \textbf{Mai} & \textbf{Jun} \\
        \hline
        Revisão bibliográfica & \checkmark & \checkmark & & & & \\
        \hline
        Elaboração dos instrumentos & & \checkmark & \checkmark & & & \\
        \hline
        Coleta de dados & & & \checkmark & \checkmark & & \\
        \hline
        Análise dos dados & & & & \checkmark & \checkmark & \\
        \hline
        Redação do relatório & & & & & \checkmark & \checkmark \\
        \hline
        Revisão e entrega & & & & & & \checkmark \\
        \hline
    \end{tabular}
    \fonte{Elaborado pelo autor.}
\end{quadro}



% ========================================================================
% ELEMENTOS PÓS-TEXTUAIS
% ========================================================================
% ========================================================================
% ELEMENTOS PÓS-TEXTUAIS
% Referências, apêndices e anexos conforme ABNT NBR 14724:2011
% ========================================================================

% ========================================================================
% REFERÊNCIAS BIBLIOGRÁFICAS
% ========================================================================
\postextual

% Configura o título das referências
\bibliography{referencias}

% ========================================================================
% GLOSSÁRIO (OPCIONAL)
% ========================================================================
% Descomente se necessário incluir glossário
% \glossary

% ========================================================================
% APÊNDICES (OPCIONAL)
% ========================================================================
% Descomente e personalize se houver apêndices
% Os apêndices são textos elaborados pelo próprio autor

% \begin{apendicesenv}
%     % Imprime uma página indicando o início dos apêndices
%     \partapendices
%     
%     % ========================================================================
%     % APÊNDICE A - EXEMPLO DE APÊNDICE
%     % ========================================================================
%     \chapter{TÍTULO DO PRIMEIRO APÊNDICE}
%     \label{apendice:primeiro}
%     
%     Este é um exemplo de apêndice. Os apêndices são textos ou documentos elaborados pelo próprio autor, a fim de complementar sua argumentação, sem prejuízo da unidade nuclear do trabalho.
%     
%     Exemplos de conteúdo para apêndices:
%     \begin{itemize}
%         \item Questionários utilizados na pesquisa
%         \item Roteiros de entrevistas
%         \item Códigos de programas desenvolvidos
%         \item Cálculos detalhados
%         \item Tabelas extensas com dados brutos
%         \item Figuras complementares
%     \end{itemize}
%     
%     % ========================================================================
%     % APÊNDICE B - EXEMPLO DE QUESTIONÁRIO
%     % ========================================================================
%     \chapter{QUESTIONÁRIO APLICADO NA PESQUISA}

% ========================================================================
% FIM DO DOCUMENTO
% ========================================================================
\end{document}
